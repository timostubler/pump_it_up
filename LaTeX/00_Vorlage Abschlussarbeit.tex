%Dokumentklasse
\documentclass[a4paper,12pt]{scrreprt}
\usepackage[left= 2.5cm,right = 2cm, bottom = 4 cm]{geometry}
%\usepackage[onehalfspacing]{setspace}
% ============= Packages =============

% Dokumentinformationen
\usepackage[
	pdftitle={Titel der Abschlussarbeit},
	pdfsubject={},
	pdfauthor={Timo Stubler},
	pdfkeywords={},	
	%Links nicht einrahmen	
	hidelinks
]{hyperref}


% Standard Packages
\usepackage[utf8]{inputenc}
\usepackage[english,ngerman]{babel}
\usepackage[T1]{fontenc}
\usepackage{graphicx, subfig}
\graphicspath{{img/}}
\usepackage{fancyhdr}
\usepackage{lmodern}
\usepackage{color}
\usepackage[numbers]{natbib}

% Immer als letztes Einbinden!
\usepackage{url}
% Immer als letztes Einbinden!

% zusätzliche Schriftzeichen der American Mathematical Society
\usepackage{amsfonts}
\usepackage[fleqn]{amsmath}

% Abbildungs-, Tabellen-, Formelnamen auf Englisch

\addto\captionsngerman{%
  \renewcommand{\figurename}{Figure}%
  \renewcommand{\contentsname}{Table of Contents}%
  \renewcommand{\tablename}{Table}%
  \renewcommand{\listtablename}{List of Tables}%
  \renewcommand{\listfigurename}{List of Figures}%
  \renewcommand{\refname}{References}%
  \renewcommand{\chaptername}{Chapter}%
  \renewcommand{\bibname}{References}
}
%nicht einrücken nach Absatz
%\setlength{\parindent}{0pt}


% ============= Kopf- und Fußzeile =============
\pagestyle{fancy}
%
\lhead{}
\chead{}
\rhead{\slshape \leftmark}
%%
\lfoot{}
\cfoot{\thepage}
\rfoot{}
%%
\renewcommand{\headrulewidth}{0.4pt}
\renewcommand{\footrulewidth}{0pt}

% ============= Package Einstellungen & Sonstiges ============= 
%Besondere Trennungen
\hyphenation{De-zi-mal-tren-nung}


% ============= Dokumentbeginn =============

\begin{document}
%Seiten ohne Kopf- und Fußzeile sowie Seitenzahl
\pagestyle{empty}

\begin{center}
\begin{tabular}{p{\textwidth}}

%\begin{center}
%\includegraphics[scale=0.75]{img/logos.png}
%\end{center}


\\

\begin{center}
\LARGE{\textsc{
Modelling and Parametric Study of a Micro-Fluidic Pump\\
}}
\end{center}

\\


\begin{center}
\large{Fakultät für angewandte Naturwissenschaften \\
und Mechatronik \\
der Hochschule München}
\end{center}

\\

\begin{center}
\textbf{\Large{Simulations-Studie}}
\end{center}


\begin{center}
im Studiengang Mikro- und Nanotechnik
\end{center}


\begin{center}
vorgelegt von
\end{center}

\begin{center}
\large{\textbf{Timo Stubler and Kristjan Axelsson}} \\
\end{center}

\begin{center}
\large{im November 2020}
\end{center}

\\

\\

\begin{center}
\begin{tabular}{lll}
\textbf{Erstprüfer:} & & Prof. Dr. A. Kersch\\
\end{tabular}
\end{center}

\end{tabular}
\end{center}

\addsec{Abstract}
\label{sec:zusammenfassung}

Hier der Abstrakte Abstrakt.

% Beendet eine Seite und erzwingt auf den nachfolgenden Seiten die Ausgabe aller Gleitobjekte (z.B. Abbildungen), die bislang definiert, aber noch nicht ausgegeben wurden. Dieser Befehl fügt, falls nötig, eine leere Seite ein, sodaß die nächste Seite nach den Gleitobjekten eine ungerade Seitennummer hat. 
\cleardoubleoddpage

% pagestyle für gesamtes Dokument aktivieren
\pagestyle{fancy}

%Inhaltsverzeichnis
\tableofcontents

%Verzeichnis aller Bilder
\listoffigures

%Verzeichnis aller Tabellen
%\listoftables

\chapter{Introduction}
\label{sec:introduction}

Story über Micro-Pumpen
Motivation
Paar Coole Pumpen Bilder aus EMFT.
Story auf Anwendungsbeispiel bezogen

Hier ein Figure Example mit Referenz.

\begin{figure}[htb]
  \centering  
	\subfloat[Raman Shift Map]{\includegraphics[width=0.30\columnwidth]{TSVarray_shift1.jpg}}
	\subfloat[Shift Scale]{\includegraphics[width=0.15\columnwidth]{TSVarray_shift2.jpg}}
	\subfloat[Stress Map]{\includegraphics[width=0.30\columnwidth]{TSVarray_stress1.jpg}}
	\subfloat[Stress Scale]{\includegraphics[width=0.15\columnwidth]{TSVarray_stress2.jpg}}
  \caption{TSV array (Cu filled) measurements of the Raman peak shift (a,b) and its equivalent stress values (c,d). \citep{yoo_multiwavelength_2014}}
  \label{fig:TSVarray}
\end{figure}






\chapter{Modellansätze}
\label{sec:modelapproaches}


\section{Elektrisches Ersatzmodell}
\label{sec:electricalmodel}

blabla

\subsection{Fluidic Model}
\label{subsec:fluidicmodel}

bla $\omega_i$. 

some equation

\begin{equation} \label{eq:energyshift}
\Delta E = \hbar (\omega_i - \omega_s)
\end{equation}


\chapter{Stand der Technik}
\label{cha:stateoftheart}

\section{Complex Models}
\label{sec:complex models}

Erweiterte Modellansätze, welche Vereinfachungen können wie abgeschafft werden, usw.


\chapter{Parametric Studies}
\label{cha:paramstudies}


\section{Reservoirdrücke}
\label{sec:reservoir}

\section{Grenzfrequenz}
\label{sec:cutofffreq}

\section{Ventilvariation}
\label{sec:ventil}



harmonic oscillator
\begin{equation} \label{eq:harmonicoszillator}
E_n = \Biggl(n+\frac{1}{2} \Biggr)\hbar\omega_{vib}
\end{equation}


\chapter{Results}
\label{sec:results}

Hier ganz viele Ergebnisplots!
hinsichtlich einzelner parameter variationen

\begin{figure}[htb]
  \centering  
  \includegraphics[scale=0.65]{img/VoigtFitsPlots_Various_LaserPowers.png}
  \caption{Voigt fits of the Stokes peak for different laser powers (coloured lines) and the respectively used data points (markers)}
  \label{fig:voigtfits}
\end{figure}



\chapter{Discussion}
\label{sec:discussion}

grenzen der modelle wiederholen, sinnhaftigkeit der parameterstudien

\section{Summary}
\label{sec:summary}

Discuss some results

\section{Outlook}
\label{sec:outlook}

Betonung auf erweiterbarkeit der modelle in zukünftigen arbeiten

%Literaturverzeichnis

\bibliography{References}
%\bibliographystyle{unsrtdin}
\bibliographystyle{unsrtnat}

\end{document}
